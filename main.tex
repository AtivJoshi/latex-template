\PassOptionsToPackage{svgnames}{xcolor} % use if there is a clash while loading xcolor
\documentclass{article}
\usepackage[a4paper,total={6.5in,10.5in}]{geometry}
% \usepackage{arxiv} % do not use with geometry
\usepackage{natbib}
\usepackage[utf8]{inputenc} % allow utf-8 input
\usepackage[T1]{fontenc}    % use 8-bit T1 fonts
\usepackage{lmodern}
\usepackage{url}            % simple URL typesetting
\usepackage{booktabs}       % professional-quality tables
\usepackage{amsfonts}       % blackboard math symbols
\usepackage{nicefrac}       % compact symbols for 1/2, etc.
\usepackage{microtype}      % microtypography
%\usepackage[english]{babel}
\usepackage{amssymb}
\usepackage{amsmath}
\usepackage{amsthm}
\usepackage{MnSymbol} % incompatible with the amssymb and amsfonts packages. It automatically loads the amsmath and textcomp packages.
\usepackage{lettrine}
\usepackage{color}
\usepackage{algorithm}
% \usepackage{algorithmicx} You don’t need to manually load the algorithmicx package, as this is done by algpseudocode
\usepackage{algpseudocode} % You can use the old algorithms with the algcompatible layout, but please use the algpseudocode layout for new algorithms.
\usepackage{bm}
\usepackage{bbm}
\usepackage{ifthen}
% \usepackage{cite}
\usepackage{graphicx}
% \usepackage{subfigure}
\usepackage{dsfont}
\usepackage{framed}
\usepackage{tikz}
% \usepackage{overpic} %LaTeX commands can be placed on the graphic at defined positions
% \usepackage[svgnames]{xcolor}
% \usepackage{apxproof} % defer proofs to the appendix
\usepackage{caption}
\usepackage{subcaption}
\usepackage[]{hyperref}       % hyperlinks
\usepackage{cleveref} % load after hyperref
\usepackage{mathtools}
\usepackage{helper}

\newboolean{DisplayComments}
\setboolean{DisplayComments}{true}
% \setboolean{DisplayComments}{false}
\DeclareRobustCommand{\aj}[1]{\ifthenelse{\boolean{DisplayComments}}{{\color{violet} (AJ: #1)}}{}}

\hypersetup{
	colorlinks=true,
	linkcolor=blue,
	urlcolor=red,
	citecolor=DarkBlue,
	linkbordercolor={0 0 1}
}

\newtheorem{theorem}{Theorem}
\newtheorem{proposition}{Proposition}
\newtheorem{definition}{Definition}
\newtheorem{lemma}{Lemma}
\newtheorem{assumption}{Assumption}

\title{Title}
\author{
   Ativ Joshi\\
   Manning College of Information \\and Computer Sciences\\
   University of Massachusetts, Amherst\\
   \texttt{atjoshi@umass.edu}
%    multiple authors
%    \and
%    Ativ Joshi\\
%    Manning College of Information \\and Computer Sciences\\
%    University of Massachusetts, Amherst\\
%    \texttt{atjoshi@umass.edu}
}
\date{}
\begin{document}
\maketitle

\section{Section}

\indent This is a section. \aj{This is a comment. Change \texttt{DisplayComments} to \texttt{false} to turn the comments off.}\\

Use \url{https://www.mathcha.io/} to generate tikz images.\\

Sample algorithm using \texttt{algpseudocode} which uses \texttt{algorithmicx} \cite[Section X]{cormen2022introduction}. 
\begin{algorithm}
    \caption{Euclid’s algorithm}
    \label{euclid}
    \begin{algorithmic}[1] % The number tells where the line numbering should start. Use 0 to remove line numbers
        % \Procedure{Euclid}{$a,b$} \Comment{The g.c.d. of a and b}
            \State \textbf{Input:} \(a,b \in \bbN\)
            \State $r\gets a \bmod b$
            \While{$r\not=0$} \Comment{We have the answer if r is 0}
                \State $a \gets b$
                \State $b \gets r$
                \State $r \gets a \bmod b$
            \EndWhile\label{euclidendwhile}
			\ForEach{1,\ldots,n}
				\State do some shit. \Comment{example of custom ForEach command}
			\EndForEach
            \State \textbf{return} $b$\Comment{The gcd is b}
        % \EndProcedure
    \end{algorithmic}
\end{algorithm}


\begin{theorem}[Miracle]
	P=NP
\end{theorem}

\nocite{*}
\bibliographystyle{unsrtnat}
\bibliography{references}
\end{document}